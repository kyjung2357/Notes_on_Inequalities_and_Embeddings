\documentclass[11pt,a4paper]{report}

\usepackage[top=0.5in, bottom=1in, left=0.8in, right=0.8in, headsep=0.in, centering]{geometry}
\usepackage{graphicx}
\usepackage{amsmath}
\usepackage{amsfonts}
\usepackage{amssymb}
\usepackage{amsthm}
\usepackage{bookmark}
\usepackage{enumitem}
\usepackage{float}

\newtheorem{theorem}{Theorem}[section]
\newtheorem{lemma}[theorem]{Lemma}
\newtheorem{proposition}[theorem]{Proposition}
\newtheorem{corollary}[theorem]{Corollary}
\theoremstyle{definition}
\newtheorem{definition}[theorem]{Definition}
\newtheorem{example}[theorem]{Example}
\newtheorem{remark}[theorem]{Remark}
\newtheorem{assumption}[theorem]{Assumption}
\newtheorem{problem}[theorem]{Problem}
\newtheorem{property}[theorem]{Property}

\def\Xint#1{\mathchoice
{\XXint\displaystyle\textstyle{#1}}%
{\XXint\textstyle\scriptstyle{#1}}%
{\XXint\scriptstyle\scriptscriptstyle{#1}}%
{\XXint\scriptscriptstyle\scriptscriptstyle{#1}}%
\!\int}
\def\XXint#1#2#3{{\setbox0=\hbox{$#1{#2#3}{\int}$}
\vcenter{\hbox{$#2#3$}}\kern-.5\wd0}}
\def\ddashint{\Xint=}
\def\dashint{\Xint-}

\begin{document}
\begin{titlepage}
	\centering
	{\scshape\LARGE PDE Notes \par}
	\vspace{1cm}
	{\scshape\Large \par}
	\vspace{1.5cm}
	{\huge\bfseries Notes on Inequalities and Embeddings\par}
	\vspace{2cm}
	{\Large\itshape Kiyuob Jung\par}
	\vspace{7cm}
	
	\begin{figure}[H] 
	\centering 
	\includegraphics[scale=0.4]{Illustration/Logo.png} 
	\end{figure}
	\vspace{2cm}

	Department of Mathematics,\par
	Kyungpook National University
	
	\vfill

% Bottom of the page
	{\large \today\par}
\end{titlepage}

\newpage

\tableofcontents

\newpage

\subsection*{Preface}

These are notes from the seminar with the authors that are heavily based on the book Lecture Notes on Functional Analysis With Applications to Linear Partial Differential Equations by Alberto Bressan, together with other sources that are mostly listed in the Bibliography. 
Usually, we conform the contents of the book as well as the numberings.

\subsection*{Abbreviation}

Since these notes is not for formal research, we almost always employ the following abbreviations:

\noindent $\diamond$ iff : if and only if

\noindent $\diamond$ s.t. : s.t.

\noindent $\diamond$ TFAE : The following are equivalent

\noindent $\diamond$ v.s. : vector space

\noindent $\diamond$ n.v.s. : normed vector space

\noindent $\diamond$ f.n.v.s. : finite dimensional normed vector space

\subsection*{Notation}

Let a set $X$. 
We employ the following notations:

\noindent $\diamond$ $\mathbb{R}^{n}$ : $n$-dimensional real Euclidean space, $\mathbb{R}=\mathbb{R}^{1}$

\noindent $\diamond$ $\mathbb{C}^{n}$ : $n$-dimensional complex space, $\mathbb{C}=\mathbb{C}^{1}$

\noindent $\diamond$ $\mathbb{K}$ : either $\mathbb{R}$ or $\mathbb{C}$.

\noindent $\diamond$ $\partial X$ : boundary of $X$.

\noindent $\diamond$ $\mathcal{C}(X) = \left\{ f:X \to \mathbb{R}  :  f~\text{is continuous (possibly unbounded) on }X \right\}$.

\noindent $\diamond$ $d(x,y)$ : Euclidean distance.

\newpage 

\chapter{Introduction} 


\chapter{Definitions and notations} 

In this note, we always denote $x=(x_1, x_2, \ldots, x_n)$ to be a point in $\mathbb{R}^{n}$.

\section{Definitions}

\begin{definition}
    Let $p>1$ and define $q \in \mathbb{R}$ by 
    \begin{equation*} 
        \frac{1}{p} + \frac{1}{q} = 1.
    \end{equation*}
    Then $p$ and $q$ are called \emph{conjugate exponents}.
\end{definition}

\begin{remark}
    A simple calculation shows the following:
    \begin{enumerate}[label=(\roman*)] 
    \rm\item $pq = p + q$,
    \rm\item $\displaystyle 1 = \frac{p+q}{pq}$,
    \rm\item $(p-1)(q-1)=1$,
    \rm\item $\displaystyle q= \frac{p}{p-1}$.
    \end{enumerate}
\end{remark}

\chapter{Basic properties}

\section{Averages of a function}

\begin{definition}
	Let $f \in L^1(U)$ with an open set $U \subset \mathbb{R}^{n}$.
	\begin{enumerate}[label=(\alph*)] 
	\rm\item An average of $f$ over set $E$ is 
	\begin{equation*} 
		\dashint_{E} f d x:=\frac{1}{\operatorname{meas}(E)} \int_{E} f(x) d x.
	\end{equation*}
	\rm\item An average of $f$ over the ball $B_r(x_0)$ is 
	\begin{equation*} 
		\dashint_{B_r(x_0)} f d x:= \frac{1}{\alpha(n) r^{n}} \int_{B_r(x_0)} f(x) d x.
	\end{equation*}
	\rm\item An average of $f$ over the sphere $\partial B_r(x_0)$ is 
	\begin{equation*} 
		\dashint_{\partial B_r(x_0)} f d S :=\frac{1}{n\alpha(n) r^{n-1}} \int_{\partial B_r(x_0)} f(x) d S.
	\end{equation*}
	\end{enumerate}
\end{definition}

\subsection{Change of variable}

\begin{theorem}
	Let $f:U \to \mathbb{R}$ with an open set $U \subset \mathbb{R}^{n}$.
	\begin{enumerate}[label=(\alph*)] 
	\rm\item $\displaystyle \int_{B_r(x_0)} f(ax + b) dx = \frac{1}{r^n}\int_{B_{ar}(x_0 + b)} f(x) dx$.
	\rm\item $\displaystyle \dashint_{B_r(x_0)} f(ax + b) dx = \dashint_{B_{ar}(x_0 + b)} f(x) dx$.
	\end{enumerate}
\end{theorem}

\begin{proof} 
	TODO 

	To show (b), let $\tilde{x} := ax + b$. 
	Then $d \tilde{x} = r^n dx$.
	Also, since $\left\vert \tilde{x} - b \right\vert = \left\vert ax \right\vert < \left\vert a \right\vert r$, we get $\tilde{x} \in B_{\left\vert a \right\vert r}(x_0 + b)$.
	Hence, 
	\begin{align*}
		\dashint_{B_r(x_0)} f(ax + b) dx &= \frac{1}{\alpha(n) r^{n}} \int_{B_r(x_0)} f(x) d x \\
		&= \dashint_{B_{ar}(x_0 + b)} f(\tilde{x}) d\tilde{x} \\
		&= \dashint_{B_{ar}(x_0 + b)} f(\tilde{x}) d\tilde{x}.
	\end{align*}
\end{proof}

\subsection{Polar coordinates}

Let $f: \mathbb{R}^{n} \rightarrow \mathbb{R}$ be continuous and summable. 
Then
\begin{enumerate}[label=(\alph*)]
\item 
$\displaystyle \int_{\mathbb{R}^{n}} f d x=\int_{0}^{\infty}\left(\int_{\partial B_r(x_{0})} f(x) d S\right) d r \quad \forall x_{0} \in \mathbb{R}^{n}$.
\item $\displaystyle \frac{d}{d r}\left(\int_{B_r(x_{0})} f(x) d x\right)=\int_{\partial B_r(x_{0})} f(x) d S$ for each $r>0$.
\end{enumerate}


\chapter{Inequalities} 

\section{Power inequalities}

\begin{theorem}
	The following statements holds.
	\begin{enumerate}[label=(\alph*)] 
		\rm\item $1 + x \leq e^{x} \quad \forall x \in \mathbb{R}$.
		\rm\item (Cauchy's inequality) \\[0.1cm]
		$\displaystyle xy \leq \frac{x^{2}}{2}+\frac{y^{2}}{2} \quad \forall x, y \in \mathbb{R}$ 
		\rm\item $\displaystyle e^{(x + y)/2} < \frac{e^{y} - e^{x}}{y - x} \quad \forall x, y \in \mathbb{R}$ with $x \neq y$.
	\end{enumerate}
	
\end{theorem}

\begin{theorem}
	The following statements holds.
	\begin{enumerate}[label=(\alph*)] 
		\rm\item $\displaystyle \left(\frac{1}{e}\right)^{\frac{1}{e}} \leq x^{x} \quad \forall x > 0$.
		\rm\item $\displaystyle x \leq x^{x^{x}} \quad \forall x > 0$.
		\rm\item $1 < x^{y}+y^{x}  \quad \forall x, y > 0$
		\rm\item $x^{y}+y^{x} \leq x^{x}+y^{y} \quad \forall x, y > 0$
		\rm\item $x^{e y}+y^{e x} \leq x^{e x}+y^{e y} \quad \forall x, y > 0$
		\rm\item $\displaystyle \frac{1-\frac{1}{x^{y}}}{y} \leq \ln (x) \leq \frac{x^{y} - 1}{y} \quad \forall x, y > 0$. The upper and lower bounds converge to $\ln (x)$ as $y \rightarrow 0$.
		\rm\item $2 < (x+y)^{z}+(x+z)^{y}+(y+z)^{x} \quad \forall x, y, z > 0$.
		\rm\item $x^{2 y}+y^{2 z}+z^{2 x} \leq x^{2 x}+y^{2 y}+z^{2 z} \quad \forall x, y, z > 0$.
		\rm\item $(x y z)^{(x+y+z) / 3} \leq x^{x} y^{y} z^{z} \quad \forall x, y, z > 0$
	\end{enumerate}
	
\end{theorem}

\begin{theorem}
	The following statements holds.
	\begin{enumerate}[label=(\alph*)] 
		\rm\item (Cauchy's inequality with $\varepsilon$)\\[0.1cm]
		$\displaystyle xy \leq \epsilon x^{2}+\frac{y^{2}}{4 \varepsilon} \quad \forall x, y>0$ , $\forall \varepsilon>0$.
	\end{enumerate}
\end{theorem}

\begin{theorem}
	The following statements holds.
	\begin{enumerate}[label=(\alph*)] 
		\rm\item $(x+y)^{p} < x^{p}+y^{p} \quad \forall x, y > 0$, $\forall p \in (0, 1)$.
		\rm\item $(x + y)^{p} \leq 2^{p - 1}\left(x^{p} + y^{p}\right) \quad \forall x, y \geq 0$, $\forall p \in [1, \infty)$.
	\end{enumerate}
\end{theorem}


\chapter{Embeddings}


\section{Sobolev Embedding}

In this section, we deal with embeddings of diverse Sobolev spaces into others.
Given a Sobolev space, it automatically belongs to certain other space, depending on the relationship between the integrability $p$ and the dimension $n$.\footnote{The regularity of domains also affects the inclusion.}
There are three cases:
\begin{align*}
    &p \in [1, n), \\
    &p = n, \\
    &p \in (n, \infty].
\end{align*}
In particular, the second case $p = n$ is called the \emph{borderline case}. 
What are we trying to obtain from the Sobolev embedding theory?
Broadly speaking, given a Sobolev space $W^{k, p}$, imbeddings of $W^{k, p}$ target two types of Banach spaces: either another Sobolev space $W^{j, q}$ or H$\ddot{\rm o}$lder spaces $C^{j, \alpha}$ for some constants $j \leq k$, $q \geq p$, and $0 \leq \alpha \leq 1$.

\subsection{The case $1 \leq p < n$}

\begin{definition}
    If $1 \leq p < n$, the \emph{Sobolev conjugate} $p^{\ast}$ of $p$ is defined by 
    \begin{equation*} 
        p^{\ast} := \frac{np}{n-p}.
    \end{equation*}
\end{definition}

\begin{remark}
    A simple calculation shows the following:
    \begin{enumerate}[label=(\roman*)] 
    \rm\item $p^{\ast} > p$,
    \rm\item $\displaystyle \frac{1}{p^{\ast}} = \frac{1}{p} - \frac{1}{n}$,
    \rm\item $p^{\ast} \to \infty$ as $p \to n$.
    \end{enumerate}
\end{remark}

\begin{theorem}
    \text{\rm (Gagliardo-Nirenberg-Sobolev inequality)}
    If $1 \leq p < n$, then there exists a constant $C$, depending only on $p$ and $n$, such that
    \begin{align*}
    \|u\|_{L^{p^*}\left(\mathbb{R}^n\right)} \leq C\|D u\|_{L^p\left(\mathbb{R}^n\right)},
    \end{align*}
    for all $u \in C_c^1\left(\mathbb{R}^n\right)$.
\end{theorem}

\begin{theorem}  
    \text{\rm (Estimates for $W^{1, p}, 1 \leq p < n$)}
    Let $U$ be a bounded, open subset of $\mathbb{R}^n$, and suppose $\partial U$ is $C^1$. 
    Assume $1 \leq p < n$ and $u \in W^{1, p}(U)$. 
    Then $u \in L^{p^*}(U)$, with the estimate
    \begin{align*}
    \|u\|_{L^{p^*}(U)} \leq C\|u\|_{W^{1, p}(U)},
    \end{align*}
    the constant $C$ depending only on $p, n$, and $U$.
\end{theorem}

\subsection{The case $p=n$}

\subsection{The case $n < p \leq \infty$}

For convenience as in the notation for the Sobolev conjugate, we write 
\begin{equation*} 
    \gamma:=1- \frac{n}{p},
\end{equation*}
whenever $n < p \leq \infty$.

\begin{theorem}
    \text{\rm (Morrey's inequality)}
    If $n < p \leq \infty$, then there exists a constant $C$, depending only on $p$ and $n$, such that
    \begin{align*}
    \|u\|_{C^{0, \gamma}\left(\mathbb{R}^n\right)} \leq C\|u\|_{W^{1, p}\left(\mathbb{R}^n\right)}
    \end{align*}
    for all $u \in C^1\left(\mathbb{R}^n\right)$.
\end{theorem}

\begin{theorem}
    \text{\rm (Estimates for $W^{1, p}$, $n<p \leq \infty$)}
    Let $U$ be a bounded, open subset of $\mathbb{R}^n$, and suppose $\partial U$ is $C^1$. Assume $n < p \leq \infty$ and $u \in W^{1, p}(U)$. 
    Then $u$ has a version $u^{\ast} \in C^{0, \gamma}(\overline{U})$ with with the estimate
    \begin{align*}
        \left\|u^*\right\|_{C^{0, \gamma}(\bar{U})} \leq C\|u\|_{W^{1, p}(U)},
    \end{align*}
    the constant $C$ depending only on $p, n$ and $U$.
\end{theorem}

\subsection{The case $p = \infty$}

\begin{theorem}
    \text{\rm (Characterization of $W^{1, \infty}$)}
    Let $U$ be open and bounded, with $\partial U$ of class $C^1$. 
    Then $u: U \rightarrow \mathbb{R}$ is Lipschitz continuous if and only if $u \in W^{1, \infty}(U)$
\end{theorem}

\end{document}