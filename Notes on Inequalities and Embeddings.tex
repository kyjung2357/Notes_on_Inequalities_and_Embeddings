\documentclass[a4paper,11pt]{article}


\usepackage[top=0.5in, bottom=1in, left=0.8in, right=0.8in, headsep=0.in, centering]{geometry}
\usepackage[utf8]{inputenc}
\usepackage{amsmath}
\usepackage{amsfonts}
\usepackage{amssymb}
\usepackage{graphicx}
\usepackage{enumitem}
\usepackage{amsthm}
\usepackage{caption} 
\usepackage{array}
\usepackage{caption}
\usepackage{amsmath}
\usepackage{kotex}

\newcommand\sbullet[1][.5]{\mathbin{\vcenter{\hbox{\scalebox{#1}{$\bullet$}}}}}
\newtheorem{theorem}{Theorem}[section]
\newtheorem{lemma}[theorem]{Lemma}
\newtheorem{proposition}[theorem]{Proposition}
\newtheorem{corollary}[theorem]{Corollary}
\newtheorem*{conjecture}{Conjecture}
\theoremstyle{definition}
\newtheorem{definition}[theorem]{Definition}
\newtheorem{example}[theorem]{Example}
\newtheorem{remark}[theorem]{Remark}
\newtheorem{assumption}[theorem]{Assumption}
\newtheorem{problem}[theorem]{Problem}
\newtheorem{property}[theorem]{Property}

\def\Xint#1{\mathchoice
{\XXint\displaystyle\textstyle{#1}}%
{\XXint\textstyle\scriptstyle{#1}}%
{\XXint\scriptstyle\scriptscriptstyle{#1}}%
{\XXint\scriptscriptstyle\scriptscriptstyle{#1}}%
\!\int}
\def\XXint#1#2#3{{\setbox0=\hbox{$#1{#2#3}{\int}$}
\vcenter{\hbox{$#2#3$}}\kern-.5\wd0}}
\def\ddashint{\Xint=}
\def\dashint{\Xint-}

\title{
Notes on Inequalities and Embedding}

\author{
Kiyuob Jung}

\begin{document}
\date{}
\maketitle



\section{Introduction}

In this note, we always denote $x=(x_1, x_2, \ldots, x_n)$ to be a point in $\mathbb{R}^{n}$.

\section{Definition}

가나다라마바사

\subsection{Definitions}

\section{Inequalities}

\section{Sobolev Embedding}

In this section, we deal with embeddings of diverse Sobolev spaces into others.
Given a Sobolev space, it automatically belongs to certain other space, depending on the relationship between the integrability $p$ and the dimension $n$.
There are three cases:
\begin{align*}
    &p \in [1, n), \\
    &p = n, \\
    &p \in (n, \infty].
\end{align*}
In particular, the second case $p = n$ is called the \emph{borderline case}. 

\subsection{The case $p \in [1, n)$}

\begin{definition}
    If $p \in [1, n)$, the \emph{Sobolev conjugate} $p^{\ast}$ of $p$ is defined by 
    \begin{equation*} 
        p^{\ast} := \frac{np}{n-p}.
    \end{equation*}
\end{definition}

\begin{remark}
    A simple calculation shows the following:
    \begin{enumerate}[label=(\roman*)] 
    \rm\item $p^{\ast} > p$,
    \rm\item $\displaystyle \frac{1}{p^{\ast}} = \frac{1}{p} - \frac{1}{n}$.
    \end{enumerate}
\end{remark}

$\Omega \subset \mathbb{R}^{n}$ to be open and 

\begin{theorem}
[Hypothesis]. The following are equivalent:
\begin{enumerate}[label=(\roman*)]
\rm \item ${\it text}$. \label{item:1st}
\rm \item ${\it text}$. \label{item:2nd}
\rm \item ${\it text}$. \label{item:3rd}
\end{enumerate}
\end{theorem}

\begin{lemma}
[Hypothesis]. The following hold:
\begin{enumerate}[label=(\alph*)]
\rm \item ${\it text}$ \label{item:first}
\rm \item ${\it text}$ \label{item:second}
\rm \item ${\it text}$ \label{item:third}
\end{enumerate}
\end{lemma}

$$
 =
\begin{cases} 
, & \text{if } \\ 
, & \text{if } \\ 
, & \text{if }  
\end{cases}
$$



then
\begin{enumerate}
\rm\item the direct scattery problem is to determine $u^s$ from $u^i$;
\rm\item the inverse scattery problem is to determine the nature of inhomogeneity to reconstruct the differential equation and/or its domain from a knowledge of the asymptotic behavior $u^s$.
\end{enumerate}





\end{document}



