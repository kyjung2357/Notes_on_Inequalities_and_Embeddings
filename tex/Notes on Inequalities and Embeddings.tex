\documentclass[11pt,a4paper]{report}

\usepackage[top=0.5in, bottom=1in, left=0.8in, right=0.8in, headsep=0.in, centering]{geometry}
\usepackage{graphicx}
\usepackage{amsmath}
\usepackage{amsfonts}
\usepackage{amssymb}
\usepackage{amsthm}
\usepackage{bookmark}
\usepackage{enumitem}
\usepackage{float}

\newtheorem{theorem}{Theorem}[section]
\newtheorem{lemma}[theorem]{Lemma}
\newtheorem{proposition}[theorem]{Proposition}
\newtheorem{corollary}[theorem]{Corollary}
\theoremstyle{definition}
\newtheorem{definition}[theorem]{Definition}
\newtheorem{example}[theorem]{Example}
\newtheorem{remark}[theorem]{Remark}
\newtheorem{assumption}[theorem]{Assumption}
\newtheorem{problem}[theorem]{Problem}
\newtheorem{property}[theorem]{Property}

\def\Xint#1{\mathchoice
{\XXint\displaystyle\textstyle{#1}}%
{\XXint\textstyle\scriptstyle{#1}}%
{\XXint\scriptstyle\scriptscriptstyle{#1}}%
{\XXint\scriptscriptstyle\scriptscriptstyle{#1}}%
\!\int}
\def\XXint#1#2#3{{\setbox0=\hbox{$#1{#2#3}{\int}$}
\vcenter{\hbox{$#2#3$}}\kern-.5\wd0}}
\def\ddashint{\Xint=}
\def\dashint{\Xint-}

\begin{document}
\begin{titlepage}
	\centering
	{\scshape\LARGE PDE Notes \par}
	\vspace{1cm}
	{\scshape\Large \par}
	\vspace{1.5cm}
	{\huge\bfseries Notes on Inequalities and Embeddings\par}
	\vspace{2cm}
	{\Large\itshape Kiyuob Jung\par}
	\vspace{2cm}
	{\Large\bfseries (in progress)\par}
	\vspace{5cm}
	
	\begin{figure}[H] 
	\centering 
	\includegraphics[scale=0.4]{Illustration/Logo.png} 
	\end{figure}
	\vspace{2cm}

	Department of Mathematics,\par
	Kyungpook National University
	
	\vfill

% Bottom of the page
	{\large \today\par}
\end{titlepage}

\newpage

\tableofcontents

\newpage

\subsection*{Preface}

Todo.


\subsection*{Abbreviation}

Since these notes is not for formal research, we almost always employ the following abbreviations:

\noindent $\diamond$ s.t. : s.t.

\noindent $\diamond$ TFAE : The following are equivalent


\subsection*{Notation}

Let a set $X$. 
We employ the following notations:

\noindent $\diamond$ $\mathbb{R}^{n}$ : $n$-dimensional real Euclidean space, $\mathbb{R}=\mathbb{R}^{1}$

\noindent $\diamond$ $\mathbb{C}^{n}$ : $n$-dimensional complex space, $\mathbb{C}=\mathbb{C}^{1}$

\noindent $\diamond$ $\mathbb{K}$ : either $\mathbb{R}$ or $\mathbb{C}$.

\noindent $\diamond$ $\partial X$ : boundary of $X$.

\noindent $\diamond$ $\forall$ : for all.

\chapter{Definitions and notations} 

In this note, we always denote $x=(x_1, x_2, \ldots, x_n)$ to be a point in $\mathbb{R}^{n}$.

\section{Definitions}

\subsection{Integration}

Here, we collect definitions for averages of a function.

\begin{definition}
	Let $f \in L^1(\Omega)$ with an open set $\Omega \subset \mathbb{R}^{n}$.
	\begin{enumerate}[label=(\alph*)] 
	\rm\item An average of $f$ over set $E$ is 
	\begin{equation*} 
		\dashint_{E} f d x:=\frac{1}{\operatorname{meas}(E)} \int_{E} f(x) d x.
	\end{equation*}
	\rm\item An average of $f$ over the ball $B_r(x_0)$ is 
	\begin{equation*} 
		\dashint_{B_r(x_0)} f d x:= \frac{1}{\alpha(n) r^{n}} \int_{B_r(x_0)} f(x) d x.
	\end{equation*}
	\rm\item An average of $f$ over the sphere $\partial B_r(x_0)$ is 
	\begin{equation*} 
		\dashint_{\partial B_r(x_0)} f d S :=\frac{1}{n\alpha(n) r^{n-1}} \int_{\partial B_r(x_0)} f(x) d S.
	\end{equation*}
	\end{enumerate}
\end{definition}

When it comes to integrability, the following definition plays an important role.

\begin{definition}
    Let $p>1$ and define $p' \in \mathbb{R}$ by 
    \begin{equation*} 
        \frac{1}{p} + \frac{1}{p'} = 1.
    \end{equation*}
    Then $p$ and $p'$ are called \emph{conjugate exponents}.
\end{definition}

Even if some authors use $q$ instead of $p'$, we stick to using $p'$ in order to save letters and will use $q$ to stand for other integrability. 

\begin{remark}
    A simple calculation shows the following:
    \begin{enumerate}[label=(\roman*)] 
    \rm\item $pp' = p + p'$,
    \rm\item $\displaystyle 1 = \frac{p+p'}{pp'}$,
    \rm\item $(p-1)(p'-1)=1$,
    \rm\item $\displaystyle p'= \frac{p}{p-1}$.
    \end{enumerate}
\end{remark}

\subsection{Convex functions}

\begin{definition}
	A function $f: \mathbb{R}^n \rightarrow \mathbb{R}$ is \emph{convex} if 
	\begin{equation*} 
		f(\lambda x+(1-\lambda) y) \leq \tau f(x)+(1-\lambda) f(y)
	\end{equation*}
	for all $x, y \in \mathbb{R}^n$ and each $0 \leq \lambda\leq 1$.
\end{definition}

\begin{remark}
	If $f$ is $C^2$, then $f$ is convex if and only if $D^2 f \geq 0$.
\end{remark}

\begin{definition}
	A $C^2$ function $f:\mathbb{R}^{n} \to \mathbb{R}$ is \emph{uniformly convex} if $D^2 g \geq \theta I$  for some constant $\theta > 0$, that is, 
	\begin{equation*} 
		\sum_{i, j=1}^n f_{x_i x_j}(x) \xi_i \xi_j \geq \theta|\xi|^2
	\end{equation*}
	for all $x, \xi \in \mathbb{R}^{n}$.
\end{definition}

\chapter{Basic properties}


\section{Change of variable}

\begin{proposition}
	Let $f:\Omega \to \mathbb{R}$ with an open set $\Omega \subset \mathbb{R}^{n}$.
	\begin{enumerate}[label=(\alph*)] 
	\rm\item $\displaystyle \int_{B_r(x_0)} f(ax + b) dx = \frac{1}{r^n}\int_{B_{ar}(x_0 + b)} f(x) dx$.
	\rm\item $\displaystyle \dashint_{B_r(x_0)} f(ax + b) dx = \dashint_{B_{ar}(x_0 + b)} f(x) dx$.
	\end{enumerate}
\end{proposition}

\begin{proof} 
	TODO 

	To show (b), let $\tilde{x} := ax + b$. 
	Then $d \tilde{x} = r^n dx$.
	Also, since $\left\vert \tilde{x} - b \right\vert = \left\vert ax \right\vert < \left\vert a \right\vert r$, we get $\tilde{x} \in B_{\left\vert a \right\vert r}(x_0 + b)$.
	Hence, 
	\begin{align*}
		\dashint_{B_r(x_0)} f(ax + b) dx &= \frac{1}{\alpha(n) r^{n}} \int_{B_r(x_0)} f(x) d x \\
		&= \dashint_{B_{ar}(x_0 + b)} f(\tilde{x}) d\tilde{x} \\
		&= \dashint_{B_{ar}(x_0 + b)} f(\tilde{x}) d\tilde{x}.
	\end{align*}
\end{proof}

\begin{remark}
	Convolution. TODO
\end{remark}

\section{Coordinates}

\subsection{Polar coordinates}

\begin{proposition}

Let $f: \mathbb{R}^{n} \rightarrow \mathbb{R}$ be continuous and summable. 
Then
\begin{enumerate}[label=(\alph*)]
\rm\item $\displaystyle \int_{\mathbb{R}^{n}} f(x) d x=\int_{0}^{\infty}\left(\int_{\partial B_r(x_{0})} f(x) d\mathcal{S}\right) d r \quad \forall x_{0} \in \mathbb{R}^{n}$.
\rm\item $\displaystyle \frac{d}{d r}\left(\int_{B_r(x_{0})} f(x) d x\right)=\int_{\partial B_r(x_{0})} f(x) d\mathcal{S} \quad \forall r>0$.
\end{enumerate}
\end{proposition}

\begin{proposition}

	Let $f: \mathbb{R}^{n} \rightarrow \mathbb{R}$ be continuous and summable. 
	Then
	\begin{enumerate}[label=(\alph*)]
	\rm\item $\displaystyle \int_{B_{\varepsilon}(0)} f(x) dx = \int_{0}^{\varepsilon} \left( \int_{\partial B_r(0)} f(x) d\mathcal{S}(x) \right) dr \quad \forall \varepsilon>0$.
	\end{enumerate}
	\end{proposition}

\chapter{Inequalities} 

\section{Scalar}

\subsection{Power inequalities}

\begin{theorem}
	The following statements hold.
	\begin{enumerate}[label=(\alph*)] 
		\rm\item $1 + x \leq e^{x} \quad \forall x \in \mathbb{R}$.
		\rm\item $\displaystyle e^{(x + y)/2} < \frac{e^{y} - e^{x}}{y - x} \quad \forall x, y \in \mathbb{R}$ with $x \neq y$.
	\end{enumerate}
\end{theorem}

\begin{theorem}
	The following statements hold.
	\begin{enumerate}[label=(\alph*)] 
		\rm\item $\displaystyle \left(\frac{1}{e}\right)^{\frac{1}{e}} \leq x^{x} \quad \forall x > 0$.
		\rm\item $\displaystyle x \leq x^{x^{x}} \quad \forall x > 0$.
		\rm\item $1 < x^{y}+y^{x}  \quad \forall x, y > 0$.
		\rm\item $x^{y}+y^{x} \leq x^{x}+y^{y} \quad \forall x, y > 0$.
		\rm\item $x^{e y}+y^{e x} \leq x^{e x}+y^{e y} \quad \forall x, y > 0$.
		\rm\item $\displaystyle \frac{1-\frac{1}{x^{y}}}{y} \leq \ln (x) \leq \frac{x^{y} - 1}{y} \quad \forall x, y > 0$. The upper and lower bounds converge to $\ln (x)$ as $y \rightarrow 0$.
		\rm\item $2 < (x+y)^{z}+(x+z)^{y}+(y+z)^{x} \quad \forall x, y, z > 0$.
		\rm\item $x^{2 y}+y^{2 z}+z^{2 x} \leq x^{2 x}+y^{2 y}+z^{2 z} \quad \forall x, y, z > 0$.
		\rm\item $(x y z)^{(x+y+z) / 3} \leq x^{x} y^{y} z^{z} \quad \forall x, y, z > 0$.
	\end{enumerate}
	
\end{theorem}

\subsection{product type}

\begin{theorem}
	The following statements hold.
	\begin{enumerate}[label=(\alph*)] 
		\rm\item (Cauchy's inequality) \\[0.1cm]
		\begin{equation*} 
			xy \leq \frac{x^{2}}{2}+\frac{y^{2}}{2} \quad \forall x, y \in \mathbb{R}.
		\end{equation*}
		\rm\item (Cauchy's inequality with $\varepsilon$)\\[0.1cm]
		\begin{equation*} 
			xy \leq \varepsilon x^{2}+\frac{y^{2}}{4 \varepsilon} \quad \forall x, y>0, \quad \forall \varepsilon>0.
		\end{equation*}
		\rm\item (Young's inequality)\\[0.1cm]
		\begin{equation*} 
			xy \leq \frac{x^p}{p} + \frac{y^{p'}}{p'} \quad \forall x, y>0, \quad  \forall 1 < p, p' < \infty,
		\end{equation*}
		where $p$ and $p'$ are conjugate exponents.
	\end{enumerate}
\end{theorem}

\subsection{summation type}

\begin{theorem}
	The following statements hold.
	\begin{enumerate}[label=(\alph*)] 
		\rm\item $(x+y)^{p} < x^{p}+y^{p} \quad \forall x, y > 0$, $\forall p \in (0, 1)$.
		\rm\item $(x + y)^{p} \leq 2^{p - 1}\left(x^{p} + y^{p}\right) \quad \forall x, y \geq 0$, $\forall p \in [1, \infty)$.
	\end{enumerate}
\end{theorem}
\section{Function}

\subsection{Convex functions}

\begin{theorem}
	\emph{(Jensen's inequality)}

	Assume $f: \mathbb{R}^m \rightarrow \mathbb{R}$ is convex and $\Omega \subset \mathbb{R}^n$ is open and bounded. 
	Let $\mathbf{u}: \Omega \rightarrow \mathbb{R}^m$ be summable. 
	Then
	\begin{equation*} 
		f\left(\dashint_{\Omega} \mathbf{u} d x \right) \leq \dashint_{\Omega} f(\mathbf{u}) dx.
	\end{equation*}
\end{theorem}

\subsection{Sobolev space}

\begin{theorem}
	Let $\Omega$ be an open set in $\mathbb{R}^{n}$.
	The following statements hold.
	\begin{enumerate}[label=(\alph*)] 
		\rm\item (H\"{o}lder's inequality) \\[0.1cm]
		If $u \in L^p(\Omega)$, $v \in L^{p'}(\Omega)$ with $1 \leq p, p' \leq \infty$, where $p$ and $p'$ are conjugate exponents, then 
		\begin{equation*} 
			\int_{\Omega} |uv|dx \leq \left\| u \right\|_{L^p(\Omega)} \left\| g \right\|_{L^{q'}(\Omega)}.
		\end{equation*}
		\rm\item (General version of H\"{o}lder's inequality) \\[0.1cm]
		If $u_i \in L^{p_i}(\Omega)$ for $i = 1, 2, \ldots, m$ with $1 \leq p_1, p_2, \ldots, p_m \leq \infty$ satisfying $\frac{1}{p_1} + \frac{1}{p_2} + \cdots + \frac{1}{p_m} = 1$, then 
		\begin{equation*} 
			\int_{\Omega} |u_1 u_2 \cdots u_m |dx \leq \prod_{i=1}^m \left\| u_i \right\|_{L^{p_i}(\Omega)}.
		\end{equation*}
		\rm\item (Minkowski's inequality)\\[0.1cm]
		If $u, v \in L^p(\Omega)$ with $1 \leq p  \leq \infty$, then 
		\begin{equation*} 
			\left\| u + v\right\|_{L^p(\Omega)} \leq \left\| u \right\|_{L^p(\Omega)} + \left\| v \right\|_{L^p(\Omega)}.
		\end{equation*}
		\rm\item (Interpolation inequality)\\[0.1cm]
		Let $1 \leq p \leq r \leq q \leq \infty$, with 
		\begin{equation*} 
			\frac{1}{r} = \frac{\theta}{p} + \frac{1 - \theta}{q}
		\end{equation*}
		for some $\theta \in [0, 1]$.
		If $u \in L^p(\Omega) \cap L^q(\Omega)$, then we also have $u \in L^r(\Omega)$ and 
		\begin{equation*} 
			\left\| u \right\|_{L^r(\Omega)} \leq \left\| u \right\|^{\theta}_{L^p(\Omega)} \left\| u \right\|^{1 - \theta}_{L^q(\Omega)}.
		\end{equation*}
	\end{enumerate}
\end{theorem}


\chapter{Embeddings}


\section{Sobolev Embedding}

In this section, we deal with embeddings of diverse Sobolev spaces into others.
Given a Sobolev space, it automatically belongs to certain other space, depending on the relationship between the integrability $p$ and the dimension $n$.\footnote{The regularity of domains also affects the inclusion.}
There are three cases:
\begin{align*}
    &p \in [1, n), \\
    &p = n, \\
    &p \in (n, \infty].
\end{align*}
In particular, the second case $p = n$ is called the \emph{borderline case}. 
What are we trying to obtain from the Sobolev embedding theory?
Broadly speaking, given a Sobolev space $W^{k, p}$, imbeddings of $W^{k, p}$ target two types of Banach spaces: either another Sobolev space $W^{j, q}$ or H$\ddot{\rm o}$lder spaces $C^{j, \alpha}$ for some constants $j \leq k$, $q \geq p$, and $0 \leq \alpha \leq 1$.

\subsection{The case $1 \leq p < n$}

\begin{definition}
    If $1 \leq p < n$, the \emph{Sobolev conjugate} $p^{\ast}$ of $p$ is defined by 
    \begin{equation*} 
        p^{\ast} := \frac{np}{n-p}.
    \end{equation*}
\end{definition}

\begin{remark}
    A simple calculation shows the following:
    \begin{enumerate}[label=(\roman*)] 
    \rm\item $p^{\ast} > p$,
    \rm\item $\displaystyle \frac{1}{p^{\ast}} = \frac{1}{p} - \frac{1}{n}$,
    \rm\item $p^{\ast} \to \infty$ as $p \to n$.
    \end{enumerate}
\end{remark}

\begin{theorem}
    \emph{(Gagliardo-Nirenberg-Sobolev inequality, \cite{2010_Evans})}

    If $1 \leq p < n$, then there exists a constant $C$, depending only on $p$ and $n$, such that
    \begin{align*}
    \|u\|_{L^{p^*}\left(\mathbb{R}^n\right)} \leq C\|D u\|_{L^p\left(\mathbb{R}^n\right)},
    \end{align*}
    for all $u \in C_c^1\left(\mathbb{R}^n\right)$.
\end{theorem}

\begin{theorem}  
    \emph{(Estimates for $W^{1, p}, 1 \leq p < n$, \cite{2010_Evans})}

    Let $U$ be a bounded, open subset of $\mathbb{R}^n$, and suppose $\partial U$ is $C^1$. 
    Assume $1 \leq p < n$ and $u \in W^{1, p}(U)$. 
    Then $u \in L^{p^*}(U)$, with the estimate
    \begin{align*}
    \|u\|_{L^{p^*}(U)} \leq C\|u\|_{W^{1, p}(U)},
    \end{align*}
    the constant $C$ depending only on $p, n$, and $U$.
\end{theorem}

\subsection{The case $p=n$}

\subsection{The case $n < p \leq \infty$}

For convenience as in the notation for H\"{o}lder exponents, we write 
\begin{equation*} 
    \gamma := 1- \frac{n}{p},
\end{equation*}
whenever $n < p \leq \infty$.

\begin{theorem}
    \emph{(Morrey's inequality, \cite{2010_Evans})}

    If $n < p \leq \infty$, then there exists a constant $C$, depending only on $p$ and $n$, such that
    \begin{align*}
    \|u\|_{C^{0, \gamma}\left(\mathbb{R}^n\right)} \leq C\|u\|_{W^{1, p}\left(\mathbb{R}^n\right)}
    \end{align*}
    for all $u \in C^1\left(\mathbb{R}^n\right)$.
\end{theorem}

\begin{theorem}
    \emph{(Estimates for $W^{1, p}$, $n<p \leq \infty$, \cite{2010_Evans})}

    Let $U$ be a bounded, open subset of $\mathbb{R}^n$, and suppose $\partial U$ is $C^1$. Assume $n < p \leq \infty$ and $u \in W^{1, p}(U)$. 
    Then $u$ has a version $u^{\ast} \in C^{0, \gamma}(\overline{U})$ with with the estimate
    \begin{align*}
        \left\|u^*\right\|_{C^{0, \gamma}(\bar{U})} \leq C\|u\|_{W^{1, p}(U)},
    \end{align*}
    the constant $C$ depending only on $p, n$ and $U$.
\end{theorem}

\subsection{The case $p = \infty$}

\begin{theorem}
	\emph{(Characterization of $W^{1, \infty}$, \cite{2010_Evans})}

	Let $U$ be open and bounded, with $\partial U$ of class $C^1$. 
    Then $u: U \rightarrow \mathbb{R}$ is Lipschitz continuous if and only if $u \in W^{1, \infty}(U)$
\end{theorem}

\subsection{Summary}


\begin{theorem}
	\emph{(Sobolev, \cite[Theorem 7.29]{2012_Giaquinta})}

	Let $u \in W_0^{1, p}\left(B_R(0)\right)$, where $B_R(0) \subset \mathbb{R}^n$. Then there are universal constants $c_1, c_2, c_3$ and $c_4$, depending on $n$, such that
	\begin{enumerate}[label=(\alph*)] 
	\rm\item if $1<p<n$ then 
	\begin{equation*} 
		\|u\|_{L^{p^*}} \leq c_1\|D u\|_{L^p},
	\end{equation*}
	\rm\item if $p = n$ then 
	\begin{equation*} 
		\dashint_{B_R(0)} \exp \left(c_2 \frac{|u|}{\|D u\|_{L^n}}\right)^{\frac{n}{n-1}} d x \leq c_3,
	\end{equation*}
	\rm\item if $p>n$ then 
	\begin{equation*} 
		\|u\|_{L^{\infty}} \leq c_4 R^{1-\frac{n}{p}}\|D u\|_{L^p}.
	\end{equation*}
	\end{enumerate}
\end{theorem}




\bibliographystyle{abbrv}
\bibliography{Bibliography/Notes_on_Inequalities_and_Embeddings}{}

\end{document}